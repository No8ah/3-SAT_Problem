\section{使用的数据结构}
    \subsection{Struct}
        \subsubsection{$Clause$}
        那么什么是``Clause''呢?将``Clause''翻译成为中文,也就是``子句''\\
    同样,我们拿一个简单的例子进行说明:

    \begin{center}
        $p_{1} \cap p_2 \cap \cdots \cap p_{l} \cap \cdots \cap p_{m}$
    \end{center}

    \begin{center}
        $clause_{l} = x_{1} \cup x_{2} \cup \overline{x_{3}}$
    \end{center}

    \noindent
    我们可以很清晰地看到:\par
        (1)$p_{1} \cap p_2 \cap \cdots \cap p_{l} \cap \cdots \cap p_{m}$是一个合取式\par
        (2)$p_{l}$实际上是合取式$p_{1} \cap p_2 \cap \cdots \cap p_{l} \cap \cdots \cap p_{m}$中的一个命题公式\par
        (3)$clause_{l}$实际上就是$p_{l}$\\
    也就是说:
    \begin{center}
        布尔表达式当中的``子句''实际上就是合取式当中的``命题公式''    
    \end{center}
    \par
    那么,从另一个角度出发,这个``Clause''里面应该有什么东西呢?我们再次观察一下上面这个例子:
    \begin{center}
        $clause_{l} = x_{1} \cup x_{2} \cup \overline{x_{3}}$
    \end{center}

    \noindent
    我们可以看到:\par
        (1)这个``Clause''当中有变量``$x_{i}$''\\
    同样,很自然地,我们会问这么一个问题:\par
        (1)这个变量``$x_{i}$''是个什么东西呢?\\
    我们这里来进行回答:\par
        (1)这个``Clause''当中的变量``$x_{i}$''叫做``子句''当中的``文字'',也就是``Clause''当中的``literal''\\
    因此,我们这里的Struct ``$Clause$''就定义的很自然了:

    \begin{lstlisting}
            struct Clause {
                int literals[3];
            };
    \end{lstlisting}

    \indent
    在这里解释一下``$literal[3]$''当中的`3',这个`3'是什么意思呢?\\
    我们这里进行回答:\par
        (1)在``问题简介''当中我们已经说了,这是一个``3-SAT问题''\\
    也就是说每个``Clause''当中只有`\textbf{3}'个变量,因此我们这里是``$literal[3]$''(当然对于``K-SAT问题''而言,我们也可以改为``$literal[k]$'',视具体情况来决定这个`k'即可)
    \subsection{Array}
        \subsubsection{$AllClauses[MAX\_CLAUSES]$}
            那么什么是``$AllClauses[MAX\_CLAUSES]$''呢?从命名的角度来看其实非常好理解:\par
            (1)将所有的``Clause''(注意是一个struct,里面存储了一个int型array)放在一个``Array''当中\ ——\ 这个``Array''就叫做``AllClauses'',换言之,我们定义了一个存放``$Clause$''的``Array'':
            \begin{center}
                $AllClauses[MAX\_CLAUSES]$\\  
            \end{center}
            \par
            同样地,在这里解释一下这个``$MAX\_CLAUSES$'':

    \begin{lstlisting}
            #define MAX_CLAUSES 1200
    \end{lstlisting}
    \par
    主要是因为题目中有提到这么一个东西:``表达式数m $\leq$ 1200'',那也就是说,我这里限定在``Array''存放的``Clause''个数不超过1200个

        \subsubsection{$assignment[MAX\_VARS+1]$}
        那么什么是``$assignment[MAX\_VARS+1]$''呢?同样,我们从命名的角度出发:\par
        (1)``$assignment[MAX\_VARS+1]$''表示对``$clause_{l}$''中的``$literal$''进行赋值的赋值方案\par
        同样,我们拿一个简单的例子(变量个数n=4)进行演示,考虑:
        \begin{center}
            $clause_{l} = x_{1} \cup x_{2} \cup \overline{x_{4}}$
        \end{center}
        \par
        那么:
        \begin{center}
            $1 \quad 0 \quad -1 \quad 1$
        \end{center}
        \par
        就是一个``assignment'',也就是一个赋值方案\\
        很自然地,我们会想:\par
        (1)变量个数n很大\par
        (2)``clause''个数m很大\\
        会怎么样呢?其实本质上也是一样的,只不过有可能赋值方案变成了:
        \begin{center}
            $1 \quad 0 \quad 1 \quad -1 \cdots \cdots$
        \end{center}
        然后我们将这个``赋值方案''存储进了``$assignment[MAX\_VARS+1]$''当中罢了\\
        在这里我们有两个细节需要进行解释:\par
        (1)为什么是``$MAX\_VARS$''\par
        (2)为什么要``$+1$''\\
        针对``为什么是`$MAX\_VARS$' '':\par
            同理,因为题目提到了``变量数n $\leq$ 300'',因此我们会有:
        \begin{lstlisting}
                #define MAX_VARS 300
        \end{lstlisting}
        针对``为什么要`$+1$' '':\par
        这是一个比较有意思的点(属于是后期的一个小优化,但确实挺折磨人的),为什么这么说呢?
        这个``$+1$''实际上是为了让``$assignment[index]$''与``$x_{index}$''的两个下标进行对应,
        我们这里用一个例子来解释一下:
        \begin{center}
            $assignment[8]$:对$x_{9}$进行assign
        \end{center}
        这么看这个例子,就会有一个比较麻烦的点出现了:
        \begin{center}
            我们很难将``$assignment\_$的下标''与``$x\_$的下标''对应起来
        \end{center}
        \noindent
        并且:
        \begin{center}
            ``$x\_$的下标''=``$assignment\_$的下标''+1
        \end{center}
        因此,我们这里引入了一个``$+1$'',就是为了让``$assignment\_$的下标''与``$x\_$的下标''对应起来:
        \begin{center}
            通过将``$assignment\_$的下标''往后移动一格实现``下标对齐''
        \end{center}
        \noindent
        这是我在实现过程中遇到的一个比较巧妙的点
        \subsubsection{$positive[MAX\_VARS + 1]$与$negative[MAX\_VARS + 1]$} 
        那么什么是``$positive[MAX\_VARS + 1]$'' 和  ``$negative[MAX\_VARS + 1]$''呢?
        因为``$MAX\_VARS + 1$''已经说明的很清楚了,
        我们这里着重说明一下这两个``Array''实现了什么功能\par
        从两个``Array''声明的位置(``$Pure \ Literal \ Elimination$'')上来看,这两个``Array''实现的功能从本质上来说是一样的,
        因此我们这里主要对``$positive[MAX\_VARS + 1]$''进行分析
    